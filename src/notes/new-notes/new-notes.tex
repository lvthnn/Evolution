\documentclass[12pt, a4paper]{article}
\usepackage[T1]{fontenc} 
\usepackage{latexsym,amssymb,amsmath}
\usepackage{graphicx}
\usepackage{hyperref}
\usepackage{enumerate}
\usepackage[most]{tcolorbox}
\usepackage{pgfplots}
\usepackage{nicefrac}
\usepackage{derivative}
\usepackage{newtxtext, newtxmath}
\usepackage{caption}
\usepackage{multicol}
\pgfplotsset{compat=1.17}
\voffset=-1.0in
\hoffset=-0.5in
\textwidth=6in
\textheight=9.0in
\setlength{\jot}{.8em}
\captionsetup{labelfont=sc}

\title{Analysis of real world eco-evolutionary data using computational modelling and statistical inference}
\author{Kári Hlynsson}
\date{University of Iceland, \\ Department of Mathematics}


\begin{document}

\renewcommand{\abstractname}{Abstract}

\maketitle

\begin{abstract}
    \noindent We present a quantitative method towards analyzing eco-evolutionary data where trait values
    $z$ are recorded over a fixed time period. Statistical methods towards deriving confidence intervals towards
    parameter estimation are discussed and we see how we can arrive at a plausible estimate for environmental difficulty
    by means of the TEST (\emph{Temporal Evolutionary Simulation Tool}) model. We briefly discuss the probabilistic inference
    regarding the mechanism of selection in light of the model.
\end{abstract}

\begin{multicols}{2}
\section*{\sc Introduction}
Under Charles Darwin's traditional theory of natural selection, a population of individuals in some locus is expected to develop
towards increased fitness with respect to their environment as time passes. In his landmark publication, \emph{On the Origin of Species},
Darwin informally defines natural selection by the following:
\begin{quote}
    [...] If variations useful to any organic being ever occur, assuredly individuals thus characterized will have the best chance of being
    preserved in the struggle for life; and from the strong principle of inheritance, these will tend to produce offspring similarly
    characterized. This principle of preservation, or the survival of the fittest, I have called natural selection.
\end{quote}
Douglas Futuyama, a prominent figure in the field of evolutionary biology, defines adaption, a term closely linked with natural selection, as 
"\emph{a characteristic that enhances the survival or reproduction or organisms that bear it, relative to alternative character states}". Fitness
is an attribute which reflects the reproductive success or survivalistic ability of an individual. In this paper, we will denote fitness by $\omega$,
where $\omega_i$ represents the corresponding fitness of some individual with an index $i$ in the population.
\par The term trait value is denoted by $z$ and represents the quantitative measure of some biochemical, morphological, physiological or anatomical structure
or construct which is subject to genetic variation. As we will see,
\end{multicols}
\end{document}